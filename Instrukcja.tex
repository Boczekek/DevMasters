\documentclass[a4paper,12pt]{article}

% Pakiety
\usepackage[utf8]{inputenc}   % Kodowanie znaków
\usepackage[T1]{fontenc}      % Kodowanie fontu
\usepackage[polish]{babel}    % Język dokumentu
\usepackage{graphicx}         % Wstawianie obrazów
\usepackage{hyperref}         % Odsyłacze
\usepackage{amsmath, amssymb} % Matematyka
\usepackage{listings}         % Wstawianie kodu
\usepackage{geometry}         % Marginesy
\usepackage{float}            % Pozycjonowanie obrazków
\geometry{margin=2.5cm}

% Ustawienia listingu (kodów)
\lstset{
  language=Python,
  basicstyle=\ttfamily\small,
  numbers=left,
  numberstyle=\tiny,
  frame=single
}

\title{Portal DevMasters \\ \large{Instrukcja obsługi}}
\author{Krystian Zając}
\date{}

\begin{document}

\maketitle
\tableofcontents
\newpage

\section{Wstęp}
Niniejsza dokumentacja opisuje funkcjonowanie projektu \textbf{DevMasters} — portalu społecznościowego dedykowanego programistom i entuzjastom tworzenia oprogramowania. Głównym celem projektu jest stworzenie aplikacji webowej zintegrowanej z relacyjną bazą danych, umożliwiającej interakcję użytkowników, publikację próśb o pomoc programistyczną oraz zarządzanie zgłoszeniami i kontami.

\section{Wymagania systemowe}

\subsection{Wymagania sprzętowe}
Aplikacja może zostać uruchomiona lokalnie przy użyciu pakietu \texttt{XAMPP}. Minimalna konfiguracja sprzętowa:
\begin{itemize}
    \item Procesor: minimum 2-rdzeniowy 2.0 GHz
    \item Pamięć RAM: co najmniej 4 GB
    \item Dysk twardy: minimum 500 MB wolnego miejsca
\end{itemize}

\subsection{Wymagania programowe i użytkownika końcowego}
\begin{itemize}
    \item System operacyjny: Windows, Linux lub macOS
    \item Zainstalowany XAMPP (z Apache, PHP, MySQL)
    \item Przeglądarka internetowa wspierająca JavaScript (Google Chrome, Firefox, Edge)
    \item Włączona obsługa JavaScript w przeglądarce
\end{itemize}

\section{Instalacja}
Aby zainstalować i uruchomić projekt lokalnie:
\begin{enumerate}
  \item Zainstaluj i uruchom środowisko \textbf{XAMPP}.
  \item Włącz serwery \textbf{Apache} i \textbf{MySQL} z panelu XAMPP.
  \item Otwórz \textbf{phpMyAdmin} i zaimportuj plik \texttt{DevM.sql}.
  \item Skopiuj folder \texttt{DevMasters} do katalogu \texttt{htdocs}.
  \item W przeglądarce przejdź do adresu: \url{http://localhost/devmasters/}
\end{enumerate}

\newpage

\section{Struktura katalogów i plików}
\begin{itemize}
  \item \texttt{DevMasters/}
  \begin{itemize}
  \item \texttt{css/}
  \begin{itemize}
      \item \texttt{login.css}
      \item \texttt{style.css}
    \end{itemize}
    \item \texttt{images/}
    \begin{itemize}
      \item \texttt{icon.png}
    \end{itemize}
    \item \texttt{js/}
    \begin{itemize}
      \item \texttt{app.js}
      \item \texttt{jquery.js}
      \item \texttt{jquery.min.js}
      \item \texttt{jquery.slim.min.js}
      \item \texttt{popper.min.js}
    \end{itemize}
    \item \texttt{ap\_accept.php}
    \item \texttt{ap\_del.php}
    \item \texttt{aplikacje.php}
    \item \texttt{aplikuj.php}
    \item \texttt{dbconfig.php}
    \item \texttt{home.php}
    \item \texttt{index.php}
    \item \texttt{login.html}
    \item \texttt{login.php}
    \item \texttt{logout.php}
    \item \texttt{prosba.html}
    \item \texttt{prosba\_del.php}
    \item \texttt{prosba\_dodaj.php}
    \item \texttt{prosba\_edit.php}
    \item \texttt{prosba\_edit\_zapisz.php}
    \item \texttt{prosby.php}
    \item \texttt{register.html}
    \item \texttt{register.php}
    \item \texttt{user\_del.php}
    \item \texttt{users.php}
    \item \texttt{zlecenia.php}
  \end{itemize}
  \item \texttt{DevM.sql}
  \item \texttt{karta-projektu.pdf}
  \item \texttt{instrukcja.pdf}
\end{itemize}

\newpage

\section{Opis działania}

\subsection{Układ strony}
Strona składa się z trzech głównych komponentów:
\begin{itemize}
  \item Pasek tytułowy z logo, przyciskiem logowania i nazwą użytkownika
  \item Pasek nawigacyjny z zakładkami
  \item Główna sekcja zawartości — dynamicznie zmieniana
\end{itemize}

\begin{figure}[H]
    \centering
    \includegraphics[width=1\linewidth]{3.png}
    \caption{Interfejs główny portalu}
\end{figure}

\newpage

\subsection{Zakładka Home}

W zakładce home wyświetlane są informacje dostosowane do aktualnie zalogowanego użytkownika oraz proste statystyki pobierane z bazy danych.\\\\
Jeśli użytkownik nie jest zalogowany wyświetlany jest kafelek powitalny z zachętą do założenia konta.\\\\
Jeśli użytkownik nie posiada rangi developer, wyswietlany jest też kafelek z zachętą do wypełnienia formularza zgłoszeniowego.

\begin{figure}[H]
    \centering
    \includegraphics[width=1\linewidth]{2.png}
    \caption{Home}
\end{figure}

\newpage

\subsection{Zakładka Prośby}

Zakładka prośby służy do składania próśb do developerów o napisanie lub poprawienie kodu. Aby złożyć taką prośbę jedynym wymaganiem jest aby użytkownik był zalogowany. Wyświetlają się tu również aktywne prośby użytkownika.

\begin{figure}[H]
    \centering
    \includegraphics[width=0.7\linewidth]{1.png}
    \caption{Prośby}
\end{figure}

\newpage

\subsubsection{Tworzenie prośby}

Po kliknięciu przycisku utwórz prośbę, użytkownik zostaje przekierowany do kreatora próśb.
Prośba składa się z tytułu, opisu oraz tagów.

\begin{figure}[H]
    \centering
    \includegraphics[width=1\linewidth]{4.png}
    \caption{Kreator próśb}
\end{figure}

\newpage

\subsubsection{Edycja prośby}

Po kliknięciu przycisku edytuj, użytkownik zostaje przekierowany do edytora próśb.
Formularz powinien być uzupełniony od treści z oryginalnej prośby, które może edytować.
Przy edycji należy na nowo wybrać tagi.

\begin{figure}[H]
    \centering
    \includegraphics[width=1\linewidth]{5.png}
    \caption{Edytor próśb}
\end{figure}

\newpage

\subsection{Ekran logowania/rejestracji}

Aby się zalogować użytkownik musi pdać login i hasło do istniejącego konta. Jest też opcja założenia nowego konta. Wymagane jest wtedy podanie adresu e-mail, a następnie ustalenie loginu oraz hasła.

\begin{figure}[H]
    \centering
    \includegraphics[width=0.7\linewidth]{10.png}
    \caption{Ekran logowania}
\end{figure}

\newpage

\subsubsection{Rangi}

Na portalu dostępne są następujące rangi użytkownika:
\begin{itemize}
    \item użytkownik - podstawowa ranga z dostępem do tworzenia próśb
    \item developer - może reagować na zlecenia (prośby)
    \item moderator - może zarządzać kontami użytkowników
    \item administrator - ma pełne uprawnienia do wszystkiego, w tym do zarządzania kontami moderatorów
\end{itemize}

\subsubsection{Konta testowe}

W celu swobodnego testowania projektu, utworzone zostały testowe konta z różnymi poziomami dostępu.\\\\

Konto użytkownika\\
Login: user\\
Hasło: user\\\\

Konto developera\\
Login: dev\\
Hasło: dev\\\\

Konto moderatora\\
Login: mod\\
Hasło: mod\\\\

Konto administratora\\
Login: admin\\
Hasło: admin\\

\newpage

\subsection{Zakładka Zlecenia}

Tutaj można przeglądać prośby złożone przez innych użytkowników. Prośby zalogowanego użytkownika są odfiltrowane i wyświetlają się tylko w zakładce Prośby.

\begin{figure}[H]
    \centering
    \includegraphics[width=1\linewidth]{6.png}
    \caption{Zlecenia}
\end{figure}

\subsection{Zakładka Zgłoszenia}

Tutaj moderatorzy i administratorzy mogą przeglądać, akceptować i odrzucać zgłoszenia na zostanie developerem.

\begin{figure}[H]
    \centering
    \includegraphics[width=1\linewidth]{7.png}
    \caption{Zgłoszenia}
\end{figure}

\newpage

\subsection{Zakładka Użytkownicy}

Tutaj moderatorzy i administratorzy mogą zarządzać kontami innych użytkowników.

\begin{figure}[H]
    \centering
    \includegraphics[width=1\linewidth]{8.png}
    \caption{Lista użytkowników}
\end{figure}

\section{Diagram bazy danych}

\begin{figure}[H]
    \centering
    \includegraphics[width=1\linewidth]{9.png}
    \caption{Diagram BD}
\end{figure}

\section{Podsumowanie}
Portal \textbf{DevMasters} realizuje założoną funkcjonalność — umożliwia interakcję między użytkownikami w kontekście pomocy programistycznej, zarządzanie kontami oraz publikację i realizację próśb. Dzięki zastosowaniu PHP, MySQL i technologii front-endowych (HTML, CSS, JS), projekt może być uruchamiany lokalnie za pomocą XAMPP i z łatwością rozwijany. 


\newpage

\listoffigures

\end{document}

