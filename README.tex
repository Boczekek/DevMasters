\documentclass[a4paper,12pt]{article}

% Pakiety
\usepackage[utf8]{inputenc}   % Kodowanie znaków
\usepackage[T1]{fontenc}      % Kodowanie fontu
\usepackage[polish]{babel}    % Język dokumentu
\usepackage{graphicx}         % Wstawianie obrazów
\usepackage{hyperref}         % Odsyłacze
\usepackage{amsmath, amssymb} % Matematyka
\usepackage{listings}         % Wstawianie kodu
\usepackage{geometry}         % Marginesy
\geometry{margin=2.5cm}

% Ustawienia listingu (kodów)
\lstset{
  language=Python,
  basicstyle=\ttfamily\small,
  numbers=left,
  numberstyle=\tiny,
  frame=single
}

\title{Portal DevMasters \\ \large{Instrukcja obsługi}}
\author{Krystian Zając}
\date{}

\begin{document}

\maketitle
\tableofcontents
\newpage

\section{Wstęp}
Dokumentacja opisuje działanie projektu DevMasters. Celem projektu jest utworzenie bazy danych oraz aplikacji webowej obsługującej ją. DevMasters to portal społecznościowy przeznaczony dla programisów i pasjonatów oprogramowania.

\section{Instalacja}
Aby zainstalować projekt, należy wykonać następujące kroki:

\begin{enumerate}
  \item Pobierz program \textbf{XAMPP}.
  \item Uruchom XAMPP jako Administrator (opcjonalne)
  \item Uruchom moduł \textbf{Apache} i \textbf{MySQL}
  \item Kliknij przycisk \textbf{Admin} przy module MySQL aby przejść do \textbf{PhpMyAdmin}
  \item Zaimportuj dołączoną bazę \textbf{DevM.sql}
  \item Umieść dołączony folder \textbf{DevMasters} w lokalizacji \texttt{XAMPP\textbackslash{}htdocs}
  \item W przeglądarce przejdź do strony projektu: \url{http://localhost/devmasters/}
\end{enumerate}

\newpage

\section{Struktura katalogów i plików}
\begin{itemize}
  \item \texttt{DevMasters/}
  \begin{itemize}
  \item \texttt{css/}
  \begin{itemize}
      \item \texttt{login.css}
      \item \texttt{style.css}
    \end{itemize}
    \item \texttt{images/}
    \begin{itemize}
      \item \texttt{icon.png}
    \end{itemize}
    \item \texttt{js/}
    \begin{itemize}
      \item \texttt{app.js}
      \item \texttt{jquery.js}
      \item \texttt{jquery.min.js}
      \item \texttt{jquery.slim.min.js}
      \item \texttt{popper.min.js}
    \end{itemize}
    \item \texttt{ap\_accept.php}
    \item \texttt{ap\_del.php}
    \item \texttt{aplikacje.php}
    \item \texttt{aplikuj.php}
    \item \texttt{dbconfig.php}
    \item \texttt{home.php}
    \item \texttt{index.php}
    \item \texttt{login.html}
    \item \texttt{login.php}
    \item \texttt{logout.php}
    \item \texttt{prosba.html}
    \item \texttt{prosba\_del.php}
    \item \texttt{prosba\_dodaj.php}
    \item \texttt{prosba\_edit.php}
    \item \texttt{prosba\_edit\_zapisz.php}
    \item \texttt{prosby.php}
    \item \texttt{register.html}
    \item \texttt{register.php}
    \item \texttt{user\_del.php}
    \item \texttt{users.php}
    \item \texttt{zlecenia.php}
  \end{itemize}
  \item \texttt{DevM.sql}
  \item \texttt{karta-projektu.pdf}
  \item \texttt{instrukcja.pdf}
\end{itemize}

\section{Opis działania}
\subsection{Układ strony}
\subsection{Zakładka Home}
\subsection{Zakładka Prośby}
\subsection{Ekran logowania/rejestracji}
\subsubsection{Rangi}
\subsubsection{Konta testowe}
\subsection{Zakładka Zlecenia}
\subsection{Zakładka Zgłoszenia}
\subsection{Zakładka Użytkownicy}

\section{Podsumowanie}
Projekt spełnia założenia. W przyszłości planowana jest...

\end{document}

